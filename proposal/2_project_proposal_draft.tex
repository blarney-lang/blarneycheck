\documentclass[UKenglish, 12pt]{article}
\usepackage{a4wide}
\usepackage{isodate}

\newcommand{\al}{$<$}
\newcommand{\ar}{$>$}

\parindent 0pt
\parskip 6pt

\begin{document}

\thispagestyle{empty}

\rightline{\large{Jonas Fiala}}
\medskip
\rightline{\large{Trinity Hall}}
\medskip
\rightline{\large{jf613}}

\vfil

\centerline{\large Part II Project Proposal}
\vspace{0.4in}
\centerline{\Large\bf Property-based Testing for Hardware}
\vspace{0.3in}
\centerline{\large{October 2019}}

\vfil

% TODO: What to put here?
{\bf Project Originator:} This is a Model Project Proposal

\vspace{0.1in}

{\bf Resources Required:} See attached Project Resource Form

\vspace{0.5in}

{\bf Project Supervisor:} {Matthew Naylor}

\vspace{0.2in}

{\bf Signature:}

\vspace{0.5in}

{\bf Director of Studies:} {Simon Moore and Hatice Gunes}

\vspace{0.2in}

{\bf Signatures:}

\vspace{0.5in}

{\bf Overseers:} {Pietro Li\'o} and {Robert Mullins}

\vspace{0.2in}

{\bf Signatures:}

\vfil
\eject


\al\emph{Some comments are included, styled in this fashion, which
won't be in the final Proposal!}\ar

\section*{Introduction and Description of the Work}

Test rigs are common when developing hardware and so many techniques have
been developed for automatic testing and generation of test cases. Property-based
testing is a useful and now commonplace idea in automatic testing.

There has been much research regarding automatic testing for software design.
\begin{description}

\item [QuickCheck] implements random testing with a given or user specified distribution~\cite{QuickCheck}

\item [SmallCheck] implements exhaustive testing up to some depth~\cite{SmallCheck}

\end{description}
Both are existing testing libraries that encourage lightweight formal specification
of software through the reward of automatic testing and small counter-examples.
More recently this idea has also been applied to hardware.
\begin{description}

\item [BlueCheck] implements random testing \`a la QuickCheck but for a HDL (Bluespec)~\cite{BlueCheck}

\end{description}

Yet few automatic testing libraries have been created for open source HDLs, so this
project will propose to bring bounded exhaustive testing to an exciting new
Haskell library for hardware description; Blarney (https://github.com/mn416/blarney).

\al\emph{What else to add? Somehow bring \cite{Koen} into this?}\ar




\section*{Resources Required}

\begin{description}

\item [Haskell:] require ghc version 8.6.1 for Blarney, but the default
apt get ghc installs version 8.0.2 (this is what is on MCS) so must
download manually.

\item [CL account:] can access required version of ghc if necessary, also to
use Bluespec/BlueCheck if necessary.

\item [Blarney:] downloaded with git clone.

\item [FPGA:] for synthesizing testing rigs.

\item [Machines in College:] similar to MCS machines, only needed if personal
hardware fails.

\item [Personal laptop and computer:]-

\begin{description}

\item [My laptop:] 2.30 GHz CPU, 8 GB RAM, 118 GB Solid-state disk for OS
and 447 GB Solid-state disk for data, Windows OS with Ubuntu VM

\item [My computer:] 3.50 GHz CPU, 16 GB RAM, 256 GB Solid-state disk for OS
and 1 TB Solid-state disk for data, Windows OS with Ubuntu VM

\item [Contingency plans:] My contingency plans against data loss are that everything
will be held under git on Github with daily checkpoints to
my Google Drive and also weekly to USB Flash Drive kept only for
that purpose.
My contingency plans against hardware/software failure are
that I can easily transition my work to the MCS/CL machines

\end{description}

\end{description}


\section*{Starting Point}

I have basic prior knowledge, from taking the \textbf{Computer Design} course in \textbf{Part IB} and the \textbf{Digital Electronics} course in \textbf{Part IA} including the practicals for both. Therefore an important part of the project will be learning to use the required tools.


\section*{Substance and Structure of the Project}

The aim of the project is to build a Haskell library for automatic testing
of modules written in Blarney. I would primarily draw inspiration from
SmallCheck~\cite{SmallCheck} and apply it to a HDL, this would mean
bounded exhaustive testing.

There are many further possibilities beyond the success criteria for this
project:
\begin{description}

\item [Synthesis] of a test rig to an FPGA and compare speeds against a simulated test rig~\cite{BlueCheck}

\item [Sequential logic] modules could be supported, testing a bounded sequence of inputs~\cite{BlueCheck}

\item [Random testing] similar to QuickCheck and BlueCheck~\cite{QuickCheck}\cite{BlueCheck}

\item [Other verification methods] (such as an SMT solver) could be created for Blarney and then compared to the bounded exhaustive testing method \al\emph{Cite?}\ar

\end{description}

\begin{thebibliography}{9}

  \bibitem {QuickCheck}
    K. Claessen and J. Hughes. \emph{QuickCheck:  A Lightweight Tool for Random Testing of Haskell Programs}, in Proceedings of ICFP’2000, (2000)

  \bibitem {Koen}
    K. Claessen. \emph{Embedded Languages for Describing and Verifying Hardware}.  PhD thesis, Chalmers University of Technology and G\"oteborg University (2001)
    
  \bibitem {SmallCheck}
    C. Runciman, M. Naylor, F. Lindblad. \emph{SmallCheck and Lazy SmallCheck: automatic exhaustive testing for small values}, in Haskell Symposium ’08. pp. 37–48. ACM (2008) 
    
  \bibitem {BlueCheck}
    M. Naylor and S. W. Moore. \emph{A Generic Synthesisable Test Bench}, in MEMOCODE 2015, pp. 128–137 (2015)
    
\end{thebibliography}

\section*{Success Criteria}

The following should be achieved:

\begin{itemize}

\item Implement a Haskell library for automatic testing of combinatorial circuits written in Blarney

\item Devise a suite of example buggy circuits

\item Measure the proportion of bugs reported and the size of counter-examples found

\end{itemize}
\al\emph{Should this be more concrete? Can also include synthesis to FPGA as a
success criteria since it will almost certainly be done?}\ar


\section*{Timetable and Milestones}

\subsection*{25th October to 21st November}

\al\emph{4 weeks * 20\% time (10hrs a week) = 40hrs}\ar

Research property-based testing for software and also hardware, making sure to
fully understand the related work. Setup and test the SmallCheck/QuickCheck
libraries for Haskell, to better understand how they work and to improve skills
with Haskell.
Learn to use Blarney, building some example circuits as a learning exercise.

Milestones: Implement small Blarney modules of combinatorial logic.  Design
structure of testing framework and type-based generators.


\subsection*{22nd November to 5th December}
\al\emph{2 weeks * 50\% time (25hrs a week) = 50hrs}\ar

Setup BlueCheck and Bluespec to test synthesizing test benches to an
FPGA.
Decide with the help of Matt on final design of the testing library, with
further literature study if necessary. Create the enumerative generators.

Milestones: Lay out structure necessary to meet the success criteria and 
be ready to work on my own over the Christmas vacation.

\subsection*{6th to 26th December (Christmas vacation)}
\al\emph{2 weeks (+ 1 weeks holiday) * 60\% time (30hrs a week) = 60hrs}\ar

Complete success criteria, namely: Implement a Haskell library for automatic
testing of combinatorial circuits written in Blarney. If any problems arise
can use second half of Christmas vacation to finish the work, at the cost of
fewer additional features. Otherwise if finished early start work on
Dissertation.

Milestones: Code satisfies the first point of the success criteria. Introduction
chapter of Dissertation mostly complete.

\subsection*{27th December to 16th January (Christmas vacation)}
\al\emph{2 weeks (+ 1 weeks holiday) * 60\% time (30hrs a week) = 60hrs}\ar

Start building additional features as able, with priority for synthesizing test
rigs to an FPGA. Write Preparation chapter of Dissertation and start work on
Implementation chapter.

Milestones: Synthesized test rig runs on FPGA. First 2 chapters of Dissertation
mostly complete and structure of Implementation chapter laid out. Some of the
proposed additional features complete.

\subsection*{17th January to 27th February}
\al\emph{6 weeks * 6\% time (3hrs a week) = 18hrs}\ar

Create progress report and presentation. 
Run the library on the example buggy circuits in preparation for Evaluation
chapter. Start of lent term is very intense so use the long time at low intensity
to make small improvements and find any bugs.

Milestones: Progress report and presentation done. Test data for Evaluation
generated.

\subsection*{28th February to 12th March}
\al\emph{2 weeks * 40\% time (20hrs a week) = 40hrs}\ar

Complete all additional features and clean up code. Project should be stable
at this point, with all changes being only minor after this point.

Milestones: Project is complete and stable to allow writing Dissertation.

\subsection*{13th March to 23rd April (Easter vacation)}
\al\emph{2 weeks * 60\% time (30hrs a week) = 60hrs}\ar

Finish up the Introduction and Preparation chapters and write Implementation
chapter. Make only small crucial changes to code (e.g. bug fixes)

Milestones: First 3 chapters of Dissertation complete. Codebase finalized,
ready for submission. 

\subsection*{13th March to 23rd April (Easter vacation)}
\al\emph{3 weeks (+ 1 week holiday) * 60\% time (30hrs a week) = 90hrs}\ar

Write Evaluation and Conclusions chapters, running any additional tests to gather
evaluation data if necessary. I can use this time to do additional work on the
project if Timetable has been moved back, otherwise for revision for final exams.

Milestones: Dissertation finished, ready for submission if necessary.

\subsection*{24th to 30th April}
\al\emph{1 week * 75\% time = 37.5hrs}\ar

Review whole project, check the Dissertation, and spend a final few
days on whatever is in greatest need of attention.

Milestones: Dissertation polished.

\al\emph{Upper bound of 455.5hrs for project, with EV of about 350hrs}\ar


\subsection*{1st to 7th May}
\al\emph{1 week * ?? = up to 50hrs}\ar

Aim to submit Dissertation a week before the deadline.

Milestone: Submission of Dissertation.

\end{document}